%GCE of WPI
%by Jiamin JIAN

\documentclass[12pt,a4paper]{ctexart}
\usepackage{CJK}
\usepackage{lipsum}
\usepackage{amsmath}
\usepackage{geometry}
\usepackage{titlesec}
\usepackage{amssymb}
\usepackage{epsfig}
\usepackage{float}
\usepackage{graphicx}
\usepackage{tabularx}
\usepackage{longtable}
\usepackage{amstext}
\usepackage{blkarray}
\usepackage{amsfonts}
\usepackage{bbm}
\usepackage{listings}
\geometry{left=2.5cm,right=2.5cm,top=2.5cm,bottom=2.5cm}

\begin{document}


\begin{center}
\textbf{ GCE January, 2019}
\vspace{8pt}

Jiamin JIAN
\end{center}

\vspace{12pt}

$\textbf{Exercise 1:}$

Let $E:= [0, 1] - S_{\mathbb{Q}} = [0, 1] \bigcap (S_{\mathbb{Q}})^{c}$ where $S_{\mathbb{Q}} := \{x \in [0, 1] | x = \frac{\sqrt{p}}{q}   \,  \text{for some} \, p, q \in \mathbb{Z}^{+} \}$. Prove or disprove: There exists a closed, uncountable subset $F \subset E$.

\vspace{8pt}

$\textbf{Solution:}$

This proposition is true. Since $S_{\mathbb{Q}}$ is a countable set, there exists a bijection between $S_{\mathbb{Q}}$ and the positive rational number in the interval $[0, 1]$, so we can enumerate the set $S_{\mathbb{Q}}$ as $\{a_{n} | n \in \mathbb{N} \}$. That is to say we have $S_{\mathbb{Q}} = \{a_{n} | n \in \mathbb{N} \}$. And then we consider the union 
$$\bigcup_{n = 1}^{+ \infty}(a_{n} - \frac{\epsilon}{2^{n}}, a_{n} + \frac{\epsilon}{2^{n}}),$$
it is an open set, we denote it as $A$, then $A = \bigcup_{n = 1}^{+ \infty}(a_{n} - \frac{\epsilon}{2^{n}}, a_{n} + \frac{\epsilon}{2^{n}})$. And when $\epsilon \to 0$, we know that $A \subset [0, 1]$ and $S_{\mathbb{Q}} \subset A$.

Since $A$ is an open set, then $[0, 1] \bigcap (A)^{c}$ is a closed set. We denote $F = [0, 1] \bigcap (A)^{c}$, since the measure of set $A$ is
\begin{equation*}
    m(A) = 2 \sum_{n = 1}^{+ \infty}  \frac{\epsilon}{2^{n}} = 2 \epsilon,
\end{equation*}
then we have $m(F) = 1 - 2 \epsilon > 0$, so, the set $F$ is uncountable. Since $F \subset E$ and it is both closed and uncountable, then the proposition is true.

For any countable set $S$, $S \subset [0, 1]$, let $E = [0, 1] - S$, we can find a closed, uncountable subset $F \subset E$, and we have the supremum of the measure of $F$ is 1.

\noindent\rule[0.25\baselineskip]{\textwidth}{0.5pt}

\vspace{8pt}
$\textbf{Exercise 2:}$

For $x$ in $[-1, 1]$ set $P_{n} (x) = c_{n} (1 - x^{2})^{n}$ where $c_{n}$ is such that $\int_{-1}^{1} P_{n} = 1.$

(i) Show that there is a positive constant $C$ such that $c_{n} \leq C \sqrt{n}$.

(ii) Let $f$ be a real valued continuous function on $[0, 1]$ such that $f(0) = f(1) = 0$. Set for $x$ in $[0, 1]$
\begin{equation*}
    f_{n}(x) = \int_{0}^{1} P_{n}(x-t) f(t) \, d t
\end{equation*}
Show that $f_{n}$ is uniformly convergence to $f$.

(iii) Let $g$ be in $L^{1}((0, 1))$. Defining $g_{n}(x) = \int_{0}^{1} P_{n} (x- t) g(t) \, d t$, is $g_{n}$ uniformly convergence to $g$ in $(0, 1)$? Does $g_{n}$ converge to $g$ in $L^{1}((0, 1))$?

\vspace{8pt}
$\textbf{Solution:}$

(i) Method 1:

Since $\int_{-1}^{1} c_{n} (1-x^{2})^{n}\, d x = 1$, then we have
\begin{equation*}
   c_{n} = \frac{1}{2 \int_{0}^{1}(1-x^{2})^{n} \, d x }.
\end{equation*}
Next we need to find a lower bound of the integral term $\int_{0}^{1}(1-x^{2})^{n} \, d x$. Since for $n > 1$,
\begin{eqnarray*}
\int_{0}^{1}(1-x^{2})^{n} \, d x &\geq& \int_{0}^{\frac{1}{\sqrt{n}}}(1-x^{2})^{n} \, d x  \\
            &\geq& \frac{1}{\sqrt{n}} (1 - \frac{1}{n})^{n},
\end{eqnarray*}
then we have $c_{n} \leq \frac{\sqrt{n}}{2 (1-\frac{1}{n})^{n}}$. We just need to find a lower bound of $(1 - \frac{1}{n})^{n}$. Since $(1 - \frac{1}{n})^{n} = 1 - C_{n}^{1} \frac{1}{n} + C_{n}^{2} \frac{1}{n^{2}} + \cdots + (-\frac{1}{n})^{n} > \frac{1}{3} - \frac{2}{6n^{2}}  > \frac{1}{4}$ as $n > 1$, then we set $C = 2$, we have $c_{n} \leq C \sqrt{n}$ for $n>1$. For $n = 1$, we get $c_{1} = \frac{3}{4} < 2$, then when $C = 2$, we have $c_{n} \leq C \sqrt{n}$ holds.

Method 2:

We change the element and define $x = \sin y$, then we have $\int_{0}^{\frac{\pi}{2}}  c_{n} \cos^{2n+1}y \, d y = \frac{1}{2}$. Since 
\begin{equation*}
   \int_{0}^{\frac{\pi}{2}} \cos^{2n+1}y \, d y = 2n \int_{0}^{\frac{\pi}{2}} \cos^{2n-1}y \, d y - 2n \int_{0}^{\frac{\pi}{2}} \cos^{2n+1}y \, d y,
\end{equation*}
we denote $I_{2n + 1} = \int_{0}^{\frac{\pi}{2}} \cos^{2n+1}y \, d y$, then we have $(2n + 1)I_{2n+1} = 2n I_{2n-1}$. Since $I_{1} = \int_{0}^{\frac{\pi}{2}} \cos y \, d y = 1$, we have $\int_{0}^{\frac{\pi}{2}} \cos^{2n+1}y \, d y = \frac{(2n)!!}{(2n+1)!!}$. And since
\begin{eqnarray*}
\frac{(2n)!!}{(2n+1)!!} &=&  \frac{2n (2n-2) \cdots 2}{(2n+1) (2n-1) \cdots 3}  \\
            &\geq& \frac{\sqrt{2n+1}\sqrt{2n-1}\sqrt{2n-1}\sqrt{2n-3} \cdots \sqrt{3}\sqrt{1}}{(2n+1) (2n-1) \cdots 3} \\
            &=& \frac{1}{\sqrt{2n+1}},
\end{eqnarray*}
then we have $c_{n} \leq \frac{\sqrt{2n+1}}{2}$. We set $C = 1$, then we have $c_{n} \leq C \sqrt{n}$.


(ii) Firstly we extend $f(x)$ to a function from $\mathbb{R}$ to $\mathbb{R}$ by zero. Then we have
\begin{equation*}
    f_{n}(x) = \int_{0}^{1} P_{n}(x-t) f(t) \, d t = \int_{\mathbb{R}}^{} P_{n}(x-t) f(t) \, d t,
\end{equation*}
then we change the element as $x - t = y$, we have
\begin{equation*}
    f_{n}(x) = \int_{\mathbb{R}}^{} P_{n}(y) f(x - y) \, d y.
\end{equation*}
Then we know that
\begin{eqnarray*}
|f_{n}(x) - f(x)| &=& \Big{|}\int_{\mathbb{R}}^{} P_{n}(y) f(x - y) \, d y - \int_{-1}^{1} P_{n}(y) f(x) \, d y \Big{|}  \\
&=& \Big{|} \int_{-1}^{1} P_{n}(y) (f(x - y) - f(x)) \, d y + \int_{([-1,1])^{c}}^{} P_{n}(y) f(x-y) \, d y \Big{|} \\
&\leq& \int_{-1}^{1} P_{n}(y) | (f(x - y) - f(x))| \, d y + \int_{([-1,1])^{c}}^{} |P_{n}(y) f(x-y)| \, d y.
\end{eqnarray*}
Since when $x \in [0, 1]$ and $y \in ([-1, 1])^{c}$, we have $x - y > 1$ or $x - y < 0$, then we have $f(x -y) = 0$, so we have
\begin{equation*}
    |f_{n}(x) - f(x)| \leq \int_{-1}^{1} P_{n}(y) | (f(x - y) - f(x))| \, d y.
\end{equation*}
And by the definition of continuous, we have $\forall \epsilon > 0$, there $\exists \delta$, when $|x - y -x| < \delta$, we have $|f(x-y) - f(x)| < \epsilon$. We denote $S = [-1,1] \bigcap [-\delta, \delta]$, since $f(x)$ is continuous in $\mathbb{R}$, we denote $\text{sup}_{x \in [0, 1]} f(x) = M$, then we have $M < + \infty$ and
\begin{eqnarray*}
|f_{n}(x) - f(x)| &\leq& \int_{-\delta}^{\delta} P_{n}(y) | (f(x - y) - f(x))| \, d y  + \int_{S}^{} P_{n}(y) | (f(x - y) - f(x))| \, d y  \\
&\leq& \epsilon \int_{-\delta}^{\delta} P_{n}(y) \, d y  + 2M \int_{S}^{} P_{n}(y) \, d y   \\
&\leq& \epsilon + 2M \int_{S}^{} c_{n} (1 - y^{2})^{n} \, d y   \\
&\leq& \epsilon + 4M C \sqrt{n} \int_{\delta}^{1} (1 - y^{2})^{n} \, d y   \\
&\leq& \epsilon + 4M C \sqrt{n} (1 - \delta)(1 - \delta^{2})^{n}.
\end{eqnarray*}
Since $\lim_{n \to + \infty} 4M C \sqrt{n} (1 - \delta)(1 - \delta^{2})^{n} = 0 $, then we can say that there exists a $N \in \mathbb{N}$, when $n > N$, we have $4M C \sqrt{n} (1 - \delta)(1 - \delta^{2})^{n} < \epsilon$. Overall, we know that $\forall x \in [0, 1], \forall \epsilon > 0$, there exists a $N \in \mathbb{N}$, when $n > N$, we have $|f_{n}(x) - f(x)| < 2 \epsilon$, so that $f_{n}$ is uniformly converges to $f$.


(iii) Firstly, the $g_{n}(x)$ is not uniformly convergent to $g$ in $(0, 1)$, we can give an counter example as following. We define 
\begin{equation*}
g(x) =
\left\{
             \begin{array}{cl}
             1, & x = \frac{1}{2} \\
             0, & x \in (0, \frac{1}{2}) \bigcup (\frac{1}{2}, 1),
             \end{array}
\right.
\end{equation*}
obviously $g(x)$ is not continuous in $(0, 1)$, but we have $g_{n} (x) = \int_{0}^{1} P_{n}(x -t)g(t) \, d t = 0, \forall x \in (0, 1)$. Then $g_{n} (x)$ is continuous in $[0, 1]$. Since $g(x)$ is not continuous in $(0, 1)$, we can say that $g_{n}(x)$ is not uniformly convergent to $g(x)$ in $(0, 1)$.

Secondly, we can show that $g_{n}(x)$ convergent to $g(x)$ in $L^{1}((0, 1))$. Since the continuous functions with compact support are dense in $L^{1}$ space, then for all $\epsilon > 0$, there exist a continuous function $f(x) \in C_{c}([0, 1])$, such that $\|f- g \|_{1} < \epsilon$. We define the $f_{n}(x)$ as the section (ii), then we have
\begin{equation*}
\|g- g_{n} \|_{1} \leq  \|g- f \|_{1} + \|f- f_{n} \|_{1} + \|f_{n}- g_{n} \|_{1}.
\end{equation*}
Since $f_{n}$ is uniformly converges to $f$, for all $\epsilon > 0$, there exists a $N \in \mathbb{N}$, when $n > N$, we have $\|f- f_{n} \|_{1} < \epsilon$. And for the same $\epsilon$, by the property that continuous function is dense in $L^{1}$ space, we have $\|f- g \|_{1} < \epsilon$. Next we verify that $\|f_{n}- g_{n} \|_{1} < \epsilon$. Since
\begin{eqnarray*}
\|f_{n}- g_{n} \|_{1} &=& \int_{0}^{1} \Big{|} \int_{0}^{1}  P_{n}(x-t) g(t) - \int_{0}^{1}  P_{n}(x-t) f(t) \, d t \Big{|} d x  \\
&=& \int_{0}^{1} \Big{|} \int_{0}^{1}  P_{n}(x-t) (g(t) - f(t)) \, d t \Big{|} d x  \\
&\leq& \int_{0}^{1}  \int_{0}^{1}  P_{n}(x-t) |g(t) - f(t)|\, d t  d x,
\end{eqnarray*}
and $P_{n}(x-t)$ is continuous for $t \in [0, 1]$, then we can find the upper bound for  $P_{n}(x-t)$, we denote it as $C$, then we have
\begin{eqnarray*}
\|f_{n}- g_{n} \|_{1} &\leq&  \int_{0}^{1}  \int_{0}^{1}  P_{n}(x-t) |g(t) - f(t)|\, d t  d x \\
&\leq&  C \int_{0}^{1}  \int_{0}^{1} |g(t) - f(t)|\, d t  d x \\
&=&  C \int_{0}^{1} |g(t) - f(t)|\, d t \\
&=& C \|g-f \|_{1}.
\end{eqnarray*}
Since $\|g-f \|_{1} < \epsilon$, we have $\|g-g_{n} \|_{1} < (2 + \frac{1}{C}) \epsilon$ for all $\epsilon > 0$. So, we know that $g_{n}(x)$ convergent to $g(x)$ in $L^{1}((0, 1))$.



\noindent\rule[0.25\baselineskip]{\textwidth}{0.5pt}

\vspace{8pt}

$\textbf{Exercise 3:}$

Give an example of $f_n, f : \mathbb{R} \mapsto [0, \infty)$ such that $f_{n} \in L^{1}(\mathbb{R})$ for every $n \in \mathbb{N}$, $f \in L^{2}(\mathbb{R})$, $f_{n} \leq f$ for every $n \in \mathbb{N}$, $f_{n} \to 0$ a.e., and $\int_{}^{} f_{n} \nrightarrow 0$.

\vspace{8pt}
$\textbf{Solution:}$

We define the $f(x) = \frac{1}{x} \mathbb{I}_{[1, +\infty)}$ and $f_{n}(x) = \frac{1}{x} \mathbb{I}_{[n, n^{2}]}$. For a fixed $n$, we have
\begin{equation*}
\int_{\mathbb{R}}^{} |f_{n}(x)| \, d x = \int_{n}^{n^{2}} \frac{1}{x} \, d x = \ln{n},
\end{equation*}
so we have $f_{n} \in L^{1}(\mathbb{R})$ for every $n \in \mathbb{N}$. And since
\begin{equation*}
\int_{\mathbb{R}}^{} |f(x)|^{2} \, d x = \int_{1}^{+ \infty} \frac{1}{x^{2}} \, d x = 1,
\end{equation*}
so we know that $f \in L^{2}(\mathbb{R})$. Since for all n, $f_{n}$ is just a part of $f$ and $f > 0$, then we have $f_{n} \leq f$ for every $n \in \mathbb{N}$. When $n \to +\infty$, we have $f_{n}(x) \leq \frac{1}{n}$, so that $f_{n} \to 0$ almost everywhere. And we calculate the integral of $f_{n}$, we have
\begin{equation*}
\int_{\mathbb{R}}^{} f_{n}(x) \, d x = \int_{n}^{n^{2}} \frac{1}{x} \, d x = \ln{n},
\end{equation*}
when $n \to + \infty$, $\ln{n} \to + \infty$, so we can get $\int_{}^{} f_{n} \nrightarrow 0$.



\end{document}
